\chapter{Introduction}\label{chapter:intro}

\lipsum[20]

\begin{wrapfigure}{r}{0.35\textwidth}
\small
\center
\begin{grammar}
<START> ::= <paren*>

<paren*> ::= <paren><paren*>
\alt $\epsilon$

<paren>  ::= \term{(} <paren*> \term{)}
\end{grammar}
\caption{A parenthesis grammar}
\label{fig:parenthesis}
\end{wrapfigure}
\lipsum[10]~\cite{Gopinath2019} \lipsum[100] (such as \Cref{fig:parenthesis}). \lipsum[100]. \Cref{fig:grammarfuzzer} shows such a tool.

A key issue with  ...

\begin{figure}
\small
\begin{lstlisting}[language=Python]
import random
def interp_key(grammar, key):
   rules = grammar[key]
   if rules:  return interp_rule(grammar, choose_one(rules)) 
   else: return [key]

def interp_rule(grammar, rule):
    return sum([interp_key(grammar, t) for t in rule], [])

\end{lstlisting}
\caption{A simple Grammar Fuzzer}
\label{fig:grammarfuzzer}
\end{figure}

This project investigates...
...\lipsum[100]

Initial results show that...
\lipsum[100]

The rest of the thesis is organized as follows: Chapter \ref{chapter:background} details the necessary background. Chapter ... Chapter~\ref{chapter:conclusion} is the conclusion.